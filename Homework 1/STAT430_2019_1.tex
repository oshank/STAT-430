\documentclass[]{article}
\usepackage{lmodern}
\usepackage{amssymb,amsmath}
\usepackage{ifxetex,ifluatex}
\usepackage{fixltx2e} % provides \textsubscript
\ifnum 0\ifxetex 1\fi\ifluatex 1\fi=0 % if pdftex
  \usepackage[T1]{fontenc}
  \usepackage[utf8]{inputenc}
\else % if luatex or xelatex
  \ifxetex
    \usepackage{mathspec}
  \else
    \usepackage{fontspec}
  \fi
  \defaultfontfeatures{Ligatures=TeX,Scale=MatchLowercase}
\fi
% use upquote if available, for straight quotes in verbatim environments
\IfFileExists{upquote.sty}{\usepackage{upquote}}{}
% use microtype if available
\IfFileExists{microtype.sty}{%
\usepackage{microtype}
\UseMicrotypeSet[protrusion]{basicmath} % disable protrusion for tt fonts
}{}
\usepackage[margin=1in]{geometry}
\usepackage{hyperref}
\hypersetup{unicode=true,
            pdfborder={0 0 0},
            breaklinks=true}
\urlstyle{same}  % don't use monospace font for urls
\usepackage{color}
\usepackage{fancyvrb}
\newcommand{\VerbBar}{|}
\newcommand{\VERB}{\Verb[commandchars=\\\{\}]}
\DefineVerbatimEnvironment{Highlighting}{Verbatim}{commandchars=\\\{\}}
% Add ',fontsize=\small' for more characters per line
\usepackage{framed}
\definecolor{shadecolor}{RGB}{248,248,248}
\newenvironment{Shaded}{\begin{snugshade}}{\end{snugshade}}
\newcommand{\KeywordTok}[1]{\textcolor[rgb]{0.13,0.29,0.53}{\textbf{#1}}}
\newcommand{\DataTypeTok}[1]{\textcolor[rgb]{0.13,0.29,0.53}{#1}}
\newcommand{\DecValTok}[1]{\textcolor[rgb]{0.00,0.00,0.81}{#1}}
\newcommand{\BaseNTok}[1]{\textcolor[rgb]{0.00,0.00,0.81}{#1}}
\newcommand{\FloatTok}[1]{\textcolor[rgb]{0.00,0.00,0.81}{#1}}
\newcommand{\ConstantTok}[1]{\textcolor[rgb]{0.00,0.00,0.00}{#1}}
\newcommand{\CharTok}[1]{\textcolor[rgb]{0.31,0.60,0.02}{#1}}
\newcommand{\SpecialCharTok}[1]{\textcolor[rgb]{0.00,0.00,0.00}{#1}}
\newcommand{\StringTok}[1]{\textcolor[rgb]{0.31,0.60,0.02}{#1}}
\newcommand{\VerbatimStringTok}[1]{\textcolor[rgb]{0.31,0.60,0.02}{#1}}
\newcommand{\SpecialStringTok}[1]{\textcolor[rgb]{0.31,0.60,0.02}{#1}}
\newcommand{\ImportTok}[1]{#1}
\newcommand{\CommentTok}[1]{\textcolor[rgb]{0.56,0.35,0.01}{\textit{#1}}}
\newcommand{\DocumentationTok}[1]{\textcolor[rgb]{0.56,0.35,0.01}{\textbf{\textit{#1}}}}
\newcommand{\AnnotationTok}[1]{\textcolor[rgb]{0.56,0.35,0.01}{\textbf{\textit{#1}}}}
\newcommand{\CommentVarTok}[1]{\textcolor[rgb]{0.56,0.35,0.01}{\textbf{\textit{#1}}}}
\newcommand{\OtherTok}[1]{\textcolor[rgb]{0.56,0.35,0.01}{#1}}
\newcommand{\FunctionTok}[1]{\textcolor[rgb]{0.00,0.00,0.00}{#1}}
\newcommand{\VariableTok}[1]{\textcolor[rgb]{0.00,0.00,0.00}{#1}}
\newcommand{\ControlFlowTok}[1]{\textcolor[rgb]{0.13,0.29,0.53}{\textbf{#1}}}
\newcommand{\OperatorTok}[1]{\textcolor[rgb]{0.81,0.36,0.00}{\textbf{#1}}}
\newcommand{\BuiltInTok}[1]{#1}
\newcommand{\ExtensionTok}[1]{#1}
\newcommand{\PreprocessorTok}[1]{\textcolor[rgb]{0.56,0.35,0.01}{\textit{#1}}}
\newcommand{\AttributeTok}[1]{\textcolor[rgb]{0.77,0.63,0.00}{#1}}
\newcommand{\RegionMarkerTok}[1]{#1}
\newcommand{\InformationTok}[1]{\textcolor[rgb]{0.56,0.35,0.01}{\textbf{\textit{#1}}}}
\newcommand{\WarningTok}[1]{\textcolor[rgb]{0.56,0.35,0.01}{\textbf{\textit{#1}}}}
\newcommand{\AlertTok}[1]{\textcolor[rgb]{0.94,0.16,0.16}{#1}}
\newcommand{\ErrorTok}[1]{\textcolor[rgb]{0.64,0.00,0.00}{\textbf{#1}}}
\newcommand{\NormalTok}[1]{#1}
\usepackage{graphicx,grffile}
\makeatletter
\def\maxwidth{\ifdim\Gin@nat@width>\linewidth\linewidth\else\Gin@nat@width\fi}
\def\maxheight{\ifdim\Gin@nat@height>\textheight\textheight\else\Gin@nat@height\fi}
\makeatother
% Scale images if necessary, so that they will not overflow the page
% margins by default, and it is still possible to overwrite the defaults
% using explicit options in \includegraphics[width, height, ...]{}
\setkeys{Gin}{width=\maxwidth,height=\maxheight,keepaspectratio}
\IfFileExists{parskip.sty}{%
\usepackage{parskip}
}{% else
\setlength{\parindent}{0pt}
\setlength{\parskip}{6pt plus 2pt minus 1pt}
}
\setlength{\emergencystretch}{3em}  % prevent overfull lines
\providecommand{\tightlist}{%
  \setlength{\itemsep}{0pt}\setlength{\parskip}{0pt}}
\setcounter{secnumdepth}{0}
% Redefines (sub)paragraphs to behave more like sections
\ifx\paragraph\undefined\else
\let\oldparagraph\paragraph
\renewcommand{\paragraph}[1]{\oldparagraph{#1}\mbox{}}
\fi
\ifx\subparagraph\undefined\else
\let\oldsubparagraph\subparagraph
\renewcommand{\subparagraph}[1]{\oldsubparagraph{#1}\mbox{}}
\fi

%%% Use protect on footnotes to avoid problems with footnotes in titles
\let\rmarkdownfootnote\footnote%
\def\footnote{\protect\rmarkdownfootnote}

%%% Change title format to be more compact
\usepackage{titling}

% Create subtitle command for use in maketitle
\newcommand{\subtitle}[1]{
  \posttitle{
    \begin{center}\large#1\end{center}
    }
}

\setlength{\droptitle}{-2em}

  \title{}
    \pretitle{\vspace{\droptitle}}
  \posttitle{}
    \author{}
    \preauthor{}\postauthor{}
    \date{}
    \predate{}\postdate{}
  

\begin{document}

\section{STAT430 Homework \#1: Due Friday, February 1,
2019.}\label{stat430-homework-1-due-friday-february-1-2019.}

\paragraph{Name: Oliver Shanklin}\label{name-oliver-shanklin}

\paragraph{Date: 1 February 2019}\label{date-1-february-2019}

\begin{center}\rule{0.5\linewidth}{\linethickness}\end{center}

\begin{enumerate}
\def\labelenumi{\arabic{enumi}.}
\setcounter{enumi}{-1}
\tightlist
\item
  This homework assignment is written in \texttt{R\ Markdown} so that
  you can open this document in \texttt{RStudio}, knit the document to
  run the \texttt{R} code provided, answer the questions, add your own
  \texttt{R} code as necessary, and turn in one final, knitted, seamless
  document with all code, numerical output, graphical output, and
  documentation in one place. Read (or re-read) Section 6.7 of Wackerly,
  Mendenhall and Schaeffer. Also read Sections 7.1-7.3.
\end{enumerate}

\textbf{Example 6.16 via simulation} This example from the book shows
that the minimum of two exponentially distributed random variables is
itself exponential. Let's explore this result via simulation. We can
simulate exponential random variables in \texttt{R} using the function
\texttt{rexp}. Take a look at the \texttt{help} for \texttt{rexp}:

\begin{Shaded}
\begin{Highlighting}[]
\KeywordTok{help}\NormalTok{(rexp)}
\end{Highlighting}
\end{Shaded}

Notice that this exponential is parameterized with \texttt{rate}, where
the rate is 1/(expected value). Let's simulate two \emph{realizations}
(simulated random values) of an exponential with
\texttt{rate\ =\ 1\ /\ 100}:

\begin{Shaded}
\begin{Highlighting}[]
\NormalTok{Y <-}\StringTok{ }\KeywordTok{rexp}\NormalTok{(}\DecValTok{2}\NormalTok{, }\DataTypeTok{rate =} \DecValTok{1} \OperatorTok{/}\StringTok{ }\DecValTok{100}\NormalTok{)}
\NormalTok{Y}
\end{Highlighting}
\end{Shaded}

\begin{verbatim}
## [1]  11.05028 380.30285
\end{verbatim}

The minimum order statistic would be

\begin{Shaded}
\begin{Highlighting}[]
\KeywordTok{min}\NormalTok{(Y)}
\end{Highlighting}
\end{Shaded}

\begin{verbatim}
## [1] 11.05028
\end{verbatim}

To conduct a simulation experiment, we want to simulate two realizations
and take their minimum---then repeat many, many times. We will use the
\texttt{apply} function to do so. To illustrate, suppose we want 5
realizations of the minimum. We'll need 10 exponentials, but let's put
them into a matrix with 2 rows and 5 columns:

\begin{Shaded}
\begin{Highlighting}[]
\NormalTok{num_realizations <-}\StringTok{ }\DecValTok{5}
\NormalTok{Y <-}\StringTok{ }\KeywordTok{rexp}\NormalTok{(}\DecValTok{2} \OperatorTok{*}\StringTok{ }\NormalTok{num_realizations, }\DataTypeTok{rate =} \DecValTok{1} \OperatorTok{/}\StringTok{ }\DecValTok{100}\NormalTok{)}
\NormalTok{Y}
\end{Highlighting}
\end{Shaded}

\begin{verbatim}
##  [1] 201.83621  26.41644  26.66566  29.34707 132.07707 110.12381  63.32088
##  [8]  15.91162  91.90949 691.00990
\end{verbatim}

\begin{Shaded}
\begin{Highlighting}[]
\NormalTok{YY <-}\StringTok{ }\KeywordTok{matrix}\NormalTok{(Y, }\DecValTok{2}\NormalTok{, num_realizations)}
\NormalTok{YY}
\end{Highlighting}
\end{Shaded}

\begin{verbatim}
##           [,1]     [,2]     [,3]     [,4]      [,5]
## [1,] 201.83621 26.66566 132.0771 63.32088  91.90949
## [2,]  26.41644 29.34707 110.1238 15.91162 691.00990
\end{verbatim}

Now we want the minimum of each column, conveniently obtained using the
\texttt{apply} function:

\begin{Shaded}
\begin{Highlighting}[]
\NormalTok{min_Y <-}\StringTok{ }\KeywordTok{apply}\NormalTok{(YY, }\DataTypeTok{MAR =} \DecValTok{2}\NormalTok{, }\DataTypeTok{FUN =} \StringTok{"min"}\NormalTok{) }\CommentTok{# 1 for rows, 2 for columns}
\NormalTok{min_Y}
\end{Highlighting}
\end{Shaded}

\begin{verbatim}
## [1]  26.41644  26.66566 110.12381  15.91162  91.90949
\end{verbatim}

Let's crank up the number of realizations to 10000. Further, let's set
our random number seed to 4302019 to make our results reproducible:

\begin{Shaded}
\begin{Highlighting}[]
\KeywordTok{set.seed}\NormalTok{(}\DecValTok{4302019}\NormalTok{)}
\NormalTok{num_realizations <-}\StringTok{ }\DecValTok{10000}
\NormalTok{Y <-}\StringTok{ }\KeywordTok{rexp}\NormalTok{(}\DecValTok{2} \OperatorTok{*}\StringTok{ }\NormalTok{num_realizations, }\DataTypeTok{rate =} \DecValTok{1} \OperatorTok{/}\StringTok{ }\DecValTok{100}\NormalTok{)}
\NormalTok{YY <-}\StringTok{ }\KeywordTok{matrix}\NormalTok{(Y, }\DecValTok{2}\NormalTok{, num_realizations)}
\NormalTok{min_Y <-}\StringTok{ }\KeywordTok{apply}\NormalTok{(YY, }\DataTypeTok{MAR =} \DecValTok{2}\NormalTok{, }\DataTypeTok{FUN =} \StringTok{"min"}\NormalTok{) }\CommentTok{# 1 for rows, 2 for columns}
\KeywordTok{hist}\NormalTok{(min_Y, }\DataTypeTok{col =} \StringTok{"LightBlue"}\NormalTok{, }\DataTypeTok{freq =} \OtherTok{FALSE}\NormalTok{) }
\end{Highlighting}
\end{Shaded}

\includegraphics{STAT430_2019_1_files/figure-latex/unnamed-chunk-6-1.pdf}

\begin{Shaded}
\begin{Highlighting}[]
\CommentTok{# freq = FALSE so that histogram is normalized to be a density, with area 1}
\end{Highlighting}
\end{Shaded}

Further, let's add the theoretical probability density function to our
histogram. First, we'll need a fine grid of \(x\)-values on which to
evaluate the function. Let's use 1000 grid points (this is quite
arbitrary as long as the number is large enough: too few and the plot
will look piecewise linear; too many is not a problem, though we won't
get visual improvement in the plot.)

\begin{Shaded}
\begin{Highlighting}[]
\NormalTok{X_grid <-}\StringTok{ }\KeywordTok{seq}\NormalTok{(}\DataTypeTok{from =} \DecValTok{0}\NormalTok{, }\DataTypeTok{to =} \KeywordTok{max}\NormalTok{(min_Y), }\DataTypeTok{length =} \DecValTok{1000}\NormalTok{)}
\NormalTok{f_X <-}\StringTok{ }\NormalTok{(}\DecValTok{1} \OperatorTok{/}\StringTok{ }\DecValTok{50}\NormalTok{) }\OperatorTok{*}\StringTok{ }\KeywordTok{exp}\NormalTok{(}\OperatorTok{-}\NormalTok{X_grid }\OperatorTok{/}\StringTok{ }\DecValTok{50}\NormalTok{)}
\KeywordTok{hist}\NormalTok{(min_Y, }\DataTypeTok{col =} \StringTok{"LightBlue"}\NormalTok{, }\DataTypeTok{freq =} \OtherTok{FALSE}\NormalTok{) }
\KeywordTok{lines}\NormalTok{(X_grid, f_X, }\DataTypeTok{lwd =} \DecValTok{3}\NormalTok{, }\DataTypeTok{col =} \StringTok{"magenta"}\NormalTok{)}
\end{Highlighting}
\end{Shaded}

\includegraphics{STAT430_2019_1_files/figure-latex/unnamed-chunk-7-1.pdf}

Finally, we can evaluate the theoretical mean and variance of the
minimum order statistic, and compare to the empirical values from our
simulation. Since the theoretical distribution is exponential with
theoretical mean equal to \(E(Y_{(1)})=50\), its variance is
\(V(Y_{(1)})=2500\). For completeness, and an example of writing up the
answer in \LaTeX, let's do those calculations analytically: The
theoretical mean of \(Y_{(1)}=\min(Y_1,Y_2)\) is \[
E(Y_{(1)})=\int_0^\infty y\frac{1}{50}e^{-y/50}\,dy=50, 
\] which is close to the simulation value of 49.55. The theoretical
second moment is \[
E(Y_{(1)}^2)=\int_0^\infty y^2\frac{1}{50} e^{-y/50}\,dy, 
\] which is a constant times the integral of a Gamma density: \[
\Gamma(3)(50)^2\int_0^\infty \frac{y^{3-1} e^{-y/50}}{\Gamma(3)(50)^3}\,dy=(2)(50)^2=5000.
\] Hence the theoretical variance is \(5000-(50)^2=2500\), which is
close to the simulation value of 2503.24.

\begin{enumerate}
\def\labelenumi{\arabic{enumi}.}
\tightlist
\item
  \textbf{Example 6.17}. Modify the code above to reproduce the analysis
  of Example 6.17, for the maximum order statistic of two exponentials,
  instead of the minimum. Reset your random number seed to 4302019, then
  (a) simulate 10000 realizations of the maximum, (b) plot their
  histogram, (c) add the theoretical probability density function (page
  335 of the textbook) to your histogram, (d) compute the empirical mean
  and variance of the maximum order statistic from your simulation.
  Finally, (e) compute the theoretical mean and variance of the maximum
  order statistic, and compare to your simulation values from (d).
\end{enumerate}

\begin{center}\rule{0.5\linewidth}{\linethickness}\end{center}

\textbf{Answer:}

Put your answer in this answer block. Some starting \texttt{R} code is
provided.

\begin{Shaded}
\begin{Highlighting}[]
\KeywordTok{set.seed}\NormalTok{(}\DecValTok{4302019}\NormalTok{)}
\NormalTok{num_realizations <-}\StringTok{ }\DecValTok{10000}
\NormalTok{### a)}


\NormalTok{Y <-}\StringTok{ }\KeywordTok{rexp}\NormalTok{(}\DecValTok{2} \OperatorTok{*}\StringTok{ }\NormalTok{num_realizations, }\DataTypeTok{rate =} \DecValTok{1} \OperatorTok{/}\StringTok{ }\DecValTok{100}\NormalTok{)}
\NormalTok{YY <-}\StringTok{ }\KeywordTok{matrix}\NormalTok{(Y, }\DecValTok{2}\NormalTok{, num_realizations)}
\NormalTok{max_Y <-}\StringTok{ }\KeywordTok{apply}\NormalTok{(YY, }\DataTypeTok{MAR =} \DecValTok{2}\NormalTok{, }\DataTypeTok{FUN =} \StringTok{"max"}\NormalTok{) }\CommentTok{# 1 for rows, 2 for columns}
\CommentTok{#hist(max_Y, col = "LightBlue", freq = FALSE)}

\NormalTok{### b) and c)}
\NormalTok{X_grid <-}\StringTok{ }\KeywordTok{seq}\NormalTok{(}\DataTypeTok{from =} \DecValTok{0}\NormalTok{, }\DataTypeTok{to =} \KeywordTok{max}\NormalTok{(max_Y), }\DataTypeTok{length =} \DecValTok{1000}\NormalTok{)}
\NormalTok{f_X <-}\StringTok{ }\NormalTok{(}\DecValTok{1} \OperatorTok{/}\StringTok{ }\DecValTok{50}\NormalTok{) }\OperatorTok{*}\StringTok{ }\NormalTok{(}\KeywordTok{exp}\NormalTok{(}\OperatorTok{-}\NormalTok{X_grid }\OperatorTok{/}\StringTok{ }\DecValTok{100}\NormalTok{) }\OperatorTok{-}\StringTok{ }\KeywordTok{exp}\NormalTok{(}\OperatorTok{-}\NormalTok{X_grid }\OperatorTok{/}\StringTok{ }\DecValTok{50}\NormalTok{))}
\KeywordTok{hist}\NormalTok{(max_Y, }\DataTypeTok{col =} \StringTok{"LightBlue"}\NormalTok{, }\DataTypeTok{freq =} \OtherTok{FALSE}\NormalTok{) }
\KeywordTok{lines}\NormalTok{(X_grid, f_X, }\DataTypeTok{lwd =} \DecValTok{3}\NormalTok{, }\DataTypeTok{col =} \StringTok{"magenta"}\NormalTok{)}
\end{Highlighting}
\end{Shaded}

\includegraphics{STAT430_2019_1_files/figure-latex/unnamed-chunk-8-1.pdf}

Parts a, b, and c are answered in the code block above.

Part d)

\begin{Shaded}
\begin{Highlighting}[]
\KeywordTok{mean}\NormalTok{(max_Y)}
\end{Highlighting}
\end{Shaded}

\begin{verbatim}
## [1] 148.8218
\end{verbatim}

\begin{Shaded}
\begin{Highlighting}[]
\KeywordTok{var}\NormalTok{(max_Y)}
\end{Highlighting}
\end{Shaded}

\begin{verbatim}
## [1] 12318.6
\end{verbatim}

Part e) This would be the equation for the expected value of the maximum
order statistic. \[
E(Y_{(n)})=\int_0^\infty y\frac{1}{50}(e^{-y/100} - e^{-y/50})\,dy = 150
\]

Here is the second moment of the maximum order statistic. \[
E(Y^2_{(n)})=\int_0^\infty y^2\frac{1}{50}(e^{-y/100} - e^{-y/50})\,dy = \frac{\Gamma(3)(100)^3}{50} - \Gamma(3)(50)^2 = 35000
\]

So the variance is,

\[
Var(Y_{(n)}) = E(Y_{(n)}^2) - \mu_{Y_{(n)}}^2 = 35000 - 150^2 = 12500
\]

The empirical mean and variance are pretty similar to the theroitical
mean and variance. Which makes sense because we had a large number of
realizations.

\begin{center}\rule{0.5\linewidth}{\linethickness}\end{center}

\begin{enumerate}
\def\labelenumi{\arabic{enumi}.}
\setcounter{enumi}{1}
\tightlist
\item
  Complete \textbf{Exercise 6.72} of the text. Assess your results by
  simulation as follows. Reset your random number seed to 4302019, then
  (a) simulate 10000 realizations of the minimum of two Uniform(0,1)
  random variables (use \texttt{runif} in \texttt{R}), (b) plot their
  histogram, (c) add the theoretical probability density function that
  you derived in 6.72(a) to your histogram, (d) compute the empirical
  mean and variance of the minimum order statistic from your simulation.
  Finally, compare the theoretical values of \(E(U_1)\) and \(V(U_1)\)
  that you computed in 6.72(b) to your simulation values from (d).
\end{enumerate}

\begin{center}\rule{0.5\linewidth}{\linethickness}\end{center}

\textbf{Answer:}

\begin{Shaded}
\begin{Highlighting}[]
\KeywordTok{set.seed}\NormalTok{(}\DecValTok{4302019}\NormalTok{)}
\NormalTok{num_realizations <-}\StringTok{ }\DecValTok{10000}


\NormalTok{Y <-}\StringTok{ }\KeywordTok{runif}\NormalTok{(}\DecValTok{2}\OperatorTok{*}\NormalTok{num_realizations, }\DecValTok{0}\NormalTok{, }\DecValTok{1}\NormalTok{)}
\NormalTok{YY <-}\StringTok{ }\KeywordTok{matrix}\NormalTok{(Y, }\DecValTok{2}\NormalTok{, num_realizations)}
\NormalTok{min_Y <-}\StringTok{ }\KeywordTok{apply}\NormalTok{(YY, }\DataTypeTok{MAR=}\DecValTok{2}\NormalTok{, }\DataTypeTok{FUN=}\StringTok{"min"}\NormalTok{)}
\KeywordTok{hist}\NormalTok{(min_Y, }\DataTypeTok{col=}\StringTok{"Pink"}\NormalTok{, }\DataTypeTok{freq=}\OtherTok{FALSE}\NormalTok{)}

\NormalTok{x_grid <-}\StringTok{ }\KeywordTok{seq}\NormalTok{(}\DataTypeTok{from =}\DecValTok{0}\NormalTok{, }\DataTypeTok{to=}\KeywordTok{max}\NormalTok{(min_Y), }\DataTypeTok{length.out =} \DecValTok{1000}\NormalTok{)}
\NormalTok{f_x <-}\StringTok{ }\DecValTok{2}\OperatorTok{*}\NormalTok{(}\DecValTok{1}\OperatorTok{-}\NormalTok{x_grid)}
\KeywordTok{lines}\NormalTok{(x_grid, f_x, }\DataTypeTok{lwd=}\DecValTok{3}\NormalTok{ ,}\DataTypeTok{col=}\StringTok{"Red"}\NormalTok{)}
\end{Highlighting}
\end{Shaded}

\includegraphics{STAT430_2019_1_files/figure-latex/unnamed-chunk-10-1.pdf}

\begin{Shaded}
\begin{Highlighting}[]
\KeywordTok{mean}\NormalTok{(min_Y)}
\end{Highlighting}
\end{Shaded}

\begin{verbatim}
## [1] 0.3350795
\end{verbatim}

\begin{Shaded}
\begin{Highlighting}[]
\KeywordTok{var}\NormalTok{(min_Y)}
\end{Highlighting}
\end{Shaded}

\begin{verbatim}
## [1] 0.05598781
\end{verbatim}

\[
E[U_{(1)}] = \int_0^1 y2(1-y)\,dy = 1/3
\] \[
E[U_{(1)}^2] = \int_0^1y^22(1-y)\,dy = 1/6
\] \[
V[U_{(1)}] = E[U_{(1)}^2] - E[U_{(1)}]^2 = 1/6 - (1/3)^2 = 1/18
\]

The empirical mean and variance are similar to the theoretical mean and
variance.

\begin{center}\rule{0.5\linewidth}{\linethickness}\end{center}

\begin{enumerate}
\def\labelenumi{\arabic{enumi}.}
\setcounter{enumi}{2}
\tightlist
\item
  Complete \textbf{Exercise 6.73} of the text. Assess your results by
  simulation as follows. Reset your random number seed to 4302019, then
  (a) simulate 10000 realizations of the maximum of two Uniform(0,1)
  random variables (use \texttt{runif} in \texttt{R}), (b) plot their
  histogram, (c) add the theoretical probability density function that
  you derived in 6.73(a) to your histogram, (d) compute the empirical
  mean and variance of the maximum order statistic from your simulation.
  Finally, compare the theoretical values of \(E(U_2)\) and \(V(U_2)\)
  that you computed in 6.73(b) to your simulation values from (d).
\end{enumerate}

\begin{center}\rule{0.5\linewidth}{\linethickness}\end{center}

\textbf{Answer:}

\begin{Shaded}
\begin{Highlighting}[]
\KeywordTok{set.seed}\NormalTok{(}\DecValTok{4302019}\NormalTok{)}
\NormalTok{num_realizations <-}\StringTok{ }\DecValTok{10000}


\NormalTok{Y <-}\StringTok{ }\KeywordTok{runif}\NormalTok{(}\DecValTok{2}\OperatorTok{*}\NormalTok{num_realizations, }\DecValTok{0}\NormalTok{, }\DecValTok{1}\NormalTok{)}
\NormalTok{YY <-}\StringTok{ }\KeywordTok{matrix}\NormalTok{(Y, }\DecValTok{2}\NormalTok{, num_realizations)}
\NormalTok{max_Y <-}\StringTok{ }\KeywordTok{apply}\NormalTok{(YY, }\DataTypeTok{MAR=}\DecValTok{2}\NormalTok{, }\DataTypeTok{FUN=}\StringTok{"max"}\NormalTok{)}
\KeywordTok{hist}\NormalTok{(max_Y, }\DataTypeTok{col=}\StringTok{"Pink"}\NormalTok{, }\DataTypeTok{freq=}\OtherTok{FALSE}\NormalTok{)}

\NormalTok{x_grid <-}\StringTok{ }\KeywordTok{seq}\NormalTok{(}\DataTypeTok{from=}\DecValTok{0}\NormalTok{, }\DataTypeTok{to =} \KeywordTok{max}\NormalTok{(max_Y), }\DataTypeTok{length.out =} \DecValTok{1000}\NormalTok{)}
\NormalTok{f_x <-}\StringTok{ }\DecValTok{2}\OperatorTok{*}\NormalTok{(x_grid)}
\KeywordTok{lines}\NormalTok{(x_grid, f_x, }\DataTypeTok{lwd=}\DecValTok{3}\NormalTok{, }\DataTypeTok{col=}\StringTok{"Red"}\NormalTok{)}
\end{Highlighting}
\end{Shaded}

\includegraphics{STAT430_2019_1_files/figure-latex/unnamed-chunk-11-1.pdf}

\begin{Shaded}
\begin{Highlighting}[]
\KeywordTok{mean}\NormalTok{(max_Y)}
\end{Highlighting}
\end{Shaded}

\begin{verbatim}
## [1] 0.6682412
\end{verbatim}

\begin{Shaded}
\begin{Highlighting}[]
\KeywordTok{var}\NormalTok{(max_Y)}
\end{Highlighting}
\end{Shaded}

\begin{verbatim}
## [1] 0.05471455
\end{verbatim}

\[
E[U_{(n)}] = \int_0^1 y2y\,dy = 2/3
\] \[
E[U_{(n)}^2] = \int_0^1 y^22y\,dy = 1/2
\] \[
Var[U_{(n)}] = E[U_{(1)}^2] - E[U_{(n)}]^2 = 1/18
\]

The empirical mean and variance are similar to the theoretical mean and
variance.

\begin{center}\rule{0.5\linewidth}{\linethickness}\end{center}

\begin{enumerate}
\def\labelenumi{\arabic{enumi}.}
\setcounter{enumi}{3}
\tightlist
\item
  Complete \textbf{Exercise 6.80} of the text. Assess your results by
  simulation for \(n=2\) as follows. Reset your random number seed to
  4302019, then (a) simulate 10000 realizations of the maximum of two
  Beta(2,2) random variables (use \texttt{rbeta} in \texttt{R} with
  \texttt{shape1\ =\ 2} and \texttt{shape2\ =\ 2}), (b) plot their
  histogram, (c) add the theoretical probability density function that
  you derived in 6.80(b) to your histogram, (d) compute the empirical
  mean and variance of the maximum order statistic from your simulation.
  Finally, compare the theoretical value of \(E(Y_{(2)}\) that you
  computed in 6.80(c) to your simulation values from (d).
\end{enumerate}

\begin{center}\rule{0.5\linewidth}{\linethickness}\end{center}

\textbf{Answer:}

\begin{Shaded}
\begin{Highlighting}[]
\KeywordTok{set.seed}\NormalTok{(}\DecValTok{4302019}\NormalTok{)}
\NormalTok{num_realizations <-}\StringTok{ }\DecValTok{10000}

\NormalTok{Y <-}\StringTok{ }\KeywordTok{rbeta}\NormalTok{(}\DecValTok{2}\OperatorTok{*}\NormalTok{num_realizations, }\DataTypeTok{shape1=}\DecValTok{2}\NormalTok{, }\DataTypeTok{shape2=}\DecValTok{2}\NormalTok{)}
\NormalTok{YY <-}\StringTok{ }\KeywordTok{matrix}\NormalTok{(Y,}\DecValTok{2}\NormalTok{,num_realizations)}
\NormalTok{max_Y <-}\StringTok{ }\KeywordTok{apply}\NormalTok{(YY, }\DataTypeTok{MAR=}\DecValTok{2}\NormalTok{, }\DataTypeTok{FUN=}\StringTok{"max"}\NormalTok{)}
\KeywordTok{hist}\NormalTok{(max_Y, }\DataTypeTok{col=}\StringTok{"Pink"}\NormalTok{, }\DataTypeTok{freq=}\OtherTok{FALSE}\NormalTok{)}

\CommentTok{# The theotetical density function}

\NormalTok{x_grid <-}\StringTok{ }\KeywordTok{seq}\NormalTok{(}\DataTypeTok{from=}\DecValTok{0}\NormalTok{, }\DataTypeTok{to =} \KeywordTok{max}\NormalTok{(max_Y), }\DataTypeTok{length.out =} \DecValTok{1000}\NormalTok{)}
\NormalTok{f_x <-}\StringTok{ }\DecValTok{12}\OperatorTok{*}\NormalTok{(x_grid)}\OperatorTok{^}\DecValTok{3}\OperatorTok{*}\NormalTok{(x_grid }\OperatorTok{-}\StringTok{ }\DecValTok{1}\NormalTok{)}\OperatorTok{*}\NormalTok{(}\DecValTok{2}\OperatorTok{*}\NormalTok{x_grid }\OperatorTok{-}\StringTok{ }\DecValTok{3}\NormalTok{) }\CommentTok{#Density function for the beta functions.}
\KeywordTok{lines}\NormalTok{(x_grid, f_x, }\DataTypeTok{lwd=}\DecValTok{3}\NormalTok{, }\DataTypeTok{col=}\StringTok{"Red"}\NormalTok{)}
\end{Highlighting}
\end{Shaded}

\includegraphics{STAT430_2019_1_files/figure-latex/unnamed-chunk-12-1.pdf}

\begin{Shaded}
\begin{Highlighting}[]
\KeywordTok{mean}\NormalTok{(max_Y)}
\end{Highlighting}
\end{Shaded}

\begin{verbatim}
## [1] 0.6290335
\end{verbatim}

\begin{Shaded}
\begin{Highlighting}[]
\KeywordTok{var}\NormalTok{(max_Y)}
\end{Highlighting}
\end{Shaded}

\begin{verbatim}
## [1] 0.0325877
\end{verbatim}

The theoretical expected value is

\[
E[Y_{(n)}] = \int_0^1 y*[12y^3(y-1)(2y-3)]dy = 22/35
\]

The theoretical second moment is \[
E[Y_{(n)}^2] = \int_0^1 y^2*[12y^3(y-1)(2y-3)]dy = 3/7
\] Varience is \[
Var[Y_{(n)}] = E[Y_{(n)}^2] - E[Y_{(n)}]^2 = 41/1225
\]

Again the theoretical expected value and varience are very similar to
the emperical solutions.

\begin{center}\rule{0.5\linewidth}{\linethickness}\end{center}

\begin{enumerate}
\def\labelenumi{\arabic{enumi}.}
\setcounter{enumi}{4}
\tightlist
\item
  In class, we used complete enumeration to study the sampling
  distribution of the sample mean when drawing an iid sample of size
  \(n=2\) from a population with \[
  P(Y_i=1)=P(Y_i=3)=P(Y_i=5)=P(Y_i=15)=1/4.
  \] The exact sampling distribution of the sample mean on the integers
  \(1, 2, \ldots, 15\) was shown to be \[
  \begin{array}{c|c|c|c|c|c|c|c|c|c|c|c|c|c|c}
  1 & 2 & 3 & 4 & 5 & 6 & 7 & 8 & 9 & 10 & 11 & 12 & 13& 14 & 15\\\hline
  \frac{1}{16}& \frac{2}{16}&\frac{3}{16}& \frac{2}{16}&\frac{1}{16}&0 & 0 & \frac{2}{16}&\frac{2}{16}& \frac{2}{16}&0 & 0 & 0 & 0 & \frac{1}{16}
  \end{array}
  \] Let's compare the exact sampling distribution to an approximation
  based on simulation:
\end{enumerate}

\begin{Shaded}
\begin{Highlighting}[]
\KeywordTok{set.seed}\NormalTok{(}\DecValTok{4302019}\NormalTok{)}
\NormalTok{num_realizations <-}\StringTok{ }\DecValTok{10000}
\NormalTok{Y <-}\StringTok{ }\KeywordTok{sample}\NormalTok{(}\KeywordTok{c}\NormalTok{(}\DecValTok{1}\NormalTok{, }\DecValTok{3}\NormalTok{, }\DecValTok{5}\NormalTok{, }\DecValTok{15}\NormalTok{), }\DataTypeTok{size =} \DecValTok{2} \OperatorTok{*}\StringTok{ }\NormalTok{num_realizations, }\DataTypeTok{replace =} \OtherTok{TRUE}\NormalTok{)}
\NormalTok{YY <-}\StringTok{ }\KeywordTok{matrix}\NormalTok{(Y, }\DecValTok{2}\NormalTok{, num_realizations)}
\NormalTok{Ybar <-}\StringTok{ }\KeywordTok{apply}\NormalTok{(YY, }\DataTypeTok{MAR =} \DecValTok{2}\NormalTok{, }\DataTypeTok{FUN =} \StringTok{"mean"}\NormalTok{)}
\CommentTok{# Compute and plot the true probabilities: }
\NormalTok{true_prob <-}\StringTok{ }\KeywordTok{c}\NormalTok{(}\DecValTok{1}\NormalTok{, }\DecValTok{2}\NormalTok{, }\DecValTok{3}\NormalTok{, }\DecValTok{2}\NormalTok{, }\DecValTok{1}\NormalTok{, }\DecValTok{0}\NormalTok{, }\DecValTok{0}\NormalTok{, }\DecValTok{2}\NormalTok{, }\DecValTok{2}\NormalTok{, }\DecValTok{2}\NormalTok{, }\DecValTok{0}\NormalTok{, }\DecValTok{0}\NormalTok{, }\DecValTok{0}\NormalTok{, }\DecValTok{0}\NormalTok{, }\DecValTok{1}\NormalTok{) }\OperatorTok{/}\StringTok{ }\DecValTok{16}
\KeywordTok{plot}\NormalTok{(}\DecValTok{1}\OperatorTok{:}\DecValTok{15}\NormalTok{, true_prob, }
     \DataTypeTok{ylab =} \StringTok{"Probability"}\NormalTok{,}
     \DataTypeTok{xlab =} \StringTok{"Sample Mean"}\NormalTok{,}
     \DataTypeTok{type =} \StringTok{"b"}\NormalTok{, }\DataTypeTok{pch =} \DecValTok{16}\NormalTok{, }\DataTypeTok{lwd =} \DecValTok{3}\NormalTok{, }\DataTypeTok{col =} \StringTok{"magenta"}\NormalTok{)}
\KeywordTok{lines}\NormalTok{(}\KeywordTok{sort}\NormalTok{(}\KeywordTok{unique}\NormalTok{(Ybar)), }\KeywordTok{table}\NormalTok{(Ybar) }\OperatorTok{/}\StringTok{ }\NormalTok{num_realizations, }\DataTypeTok{type =} \StringTok{"h"}\NormalTok{, }\DataTypeTok{lwd =} \DecValTok{3}\NormalTok{, }\DataTypeTok{col =} \StringTok{"blue"}\NormalTok{)}
\end{Highlighting}
\end{Shaded}

\includegraphics{STAT430_2019_1_files/figure-latex/unnamed-chunk-13-1.pdf}

Now adapt this example to study the \emph{range}, \[
|Y_1-Y_2|=\max\{Y_1, Y_2\} - \min\{Y_1,Y_2\}.
\] Use complete enumeration to give the exact sampling distribution of
the range. Then approximate the exact sampling distribution of the range
by simulation, drawing 10,000 samples of size \(n=2\) from the
population. Compare your exact distribution to your simulation-based
approximation with a figure like that given above.

\begin{center}\rule{0.5\linewidth}{\linethickness}\end{center}

\textbf{Answer:}

\begin{Shaded}
\begin{Highlighting}[]
\KeywordTok{set.seed}\NormalTok{(}\DecValTok{4302019}\NormalTok{)}
\NormalTok{num_realizations <-}\StringTok{ }\DecValTok{10000}
\NormalTok{Y <-}\StringTok{ }\KeywordTok{sample}\NormalTok{(}\KeywordTok{c}\NormalTok{(}\DecValTok{1}\NormalTok{, }\DecValTok{3}\NormalTok{, }\DecValTok{5}\NormalTok{, }\DecValTok{15}\NormalTok{), }\DataTypeTok{size =} \DecValTok{2} \OperatorTok{*}\StringTok{ }\NormalTok{num_realizations, }\DataTypeTok{replace =} \OtherTok{TRUE}\NormalTok{)}
\NormalTok{YY <-}\StringTok{ }\KeywordTok{matrix}\NormalTok{(Y, }\DecValTok{2}\NormalTok{, num_realizations)}


\NormalTok{Y_min <-}\StringTok{ }\KeywordTok{apply}\NormalTok{(YY, }\DataTypeTok{MAR =} \DecValTok{2}\NormalTok{, }\DataTypeTok{FUN =} \StringTok{"min"}\NormalTok{)}
\NormalTok{Y_max <-}\StringTok{ }\KeywordTok{apply}\NormalTok{(YY, }\DataTypeTok{MAR =} \DecValTok{2}\NormalTok{, }\DataTypeTok{FUN =} \StringTok{"max"}\NormalTok{)}
\NormalTok{Y_range <-}\StringTok{ }\NormalTok{Y_max }\OperatorTok{-}\StringTok{ }\NormalTok{Y_min}

\CommentTok{# Compute and plot the true probabilities: }
\NormalTok{true_prob <-}\StringTok{ }\KeywordTok{c}\NormalTok{(}\DecValTok{4}\NormalTok{, }\DecValTok{0}\NormalTok{, }\DecValTok{4}\NormalTok{, }\DecValTok{0}\NormalTok{, }\DecValTok{2}\NormalTok{, }\DecValTok{0}\NormalTok{, }\DecValTok{0}\NormalTok{, }\DecValTok{0}\NormalTok{, }\DecValTok{0}\NormalTok{, }\DecValTok{0}\NormalTok{, }\DecValTok{2}\NormalTok{, }\DecValTok{0}\NormalTok{, }\DecValTok{2}\NormalTok{, }\DecValTok{0}\NormalTok{, }\DecValTok{2}\NormalTok{) }\OperatorTok{/}\StringTok{ }\DecValTok{16}
\KeywordTok{plot}\NormalTok{(}\DecValTok{0}\OperatorTok{:}\DecValTok{14}\NormalTok{, true_prob, }
     \DataTypeTok{ylab =} \StringTok{"Probability"}\NormalTok{,}
     \DataTypeTok{xlab =} \StringTok{"Sample Range"}\NormalTok{,}
     \DataTypeTok{type =} \StringTok{"b"}\NormalTok{, }\DataTypeTok{pch =} \DecValTok{16}\NormalTok{, }\DataTypeTok{lwd =} \DecValTok{3}\NormalTok{, }\DataTypeTok{col =} \StringTok{"magenta"}\NormalTok{)}
\KeywordTok{lines}\NormalTok{(}\KeywordTok{sort}\NormalTok{(}\KeywordTok{unique}\NormalTok{(Y_range)), }\KeywordTok{table}\NormalTok{(Y_range) }\OperatorTok{/}\StringTok{ }\NormalTok{num_realizations, }\DataTypeTok{type =} \StringTok{"h"}\NormalTok{, }\DataTypeTok{lwd =} \DecValTok{3}\NormalTok{, }\DataTypeTok{col =} \StringTok{"blue"}\NormalTok{)}
\end{Highlighting}
\end{Shaded}

\includegraphics{STAT430_2019_1_files/figure-latex/unnamed-chunk-14-1.pdf}

The pmf of the \[range( Y_1, Y_2)\]

\[
\begin{array}{c|c|c|c|c|c|c|c|c|c|c|c|c|c|c}
0 & 1 & 2 & 3 & 4 & 5 & 6 & 7 & 8 & 9 & 10 & 11 & 12 & 13 & 14\\\hline
\frac{4}{16}& 0&\frac{4}{16}&0&\frac{2}{16}&0 & 0 & 0&0& 0&\frac{2}{16}&0 & \frac{2}{16} & 0 & \frac{2}{16} 
\end{array}
\]

\begin{center}\rule{0.5\linewidth}{\linethickness}\end{center}


\end{document}
